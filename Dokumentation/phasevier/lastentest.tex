\chapter{Lasttest}

\section{Testumgebung} 
Um bestimmte Anforderungen der {\grqq Nichtfunktionalen Anforderungen\grqq} zu verifizieren wurde ein Lasttest erstellt und durchgeführt. 
Dieser wurde auf einem zusätzlich dafür eingerichteten Testserver ausgeführt, welcher Skripte enthält, die es ermöglichen, die Datenbank auf einen Ursprungszustand zurückzusetzen.

\subsection{Verwaltungsaufwand N1}
\textbf{Status \textcolor{green}{Der Test läuft durch}}
\begin{itemize}
\item Voraussetzung: Server ist deployed und mit Anfangsdatenbank eingerichtet.
\item Aktion: Es werden 450 User und für jeden User zusätzlich Rezepte mit unterschiedlichem Titel und Rezeptbild erstellt, sodass insgesamt 14.000 Rezepte erstellt werden.
\item Erwartetes Ergebnis: Der Server lässt dies zu und ist auch noch mit über 14.000 Rezepten flüssig zu bedienen. 
\end{itemize}

Eine der Ergebnisse des Lastentests ist, dass die Verwaltung von 14.000 Rezepten erhöhte Speicheranforderungen mit sich bringt. In der Standardausführung der BawüCloud sind die Serverimages 12Gigabyte gross. 
Das Speichern von 14.000 Rezepten bedeutete dass Platz für 23 Gigabyte Bilderdaten nötig war. 
Durch Einhängen einer zweiten neu angelegten Festplatte (Insgesamt sind in der Standard-BWCloud-version 50 Gigabyte verfügbar) konnte dieser bereitgestellt werden. 


\section{Nichtfunktionale Anforderungen}
Weitere Nichtfunktionale Anforderungen sind durch bereit implementierte Tests überdeckt, oder nicht testbar, da die zugrundeliegenden Funktionen nicht implementiert sind.
Überdeckt sind:
\begin{itemize}[nosep]
	\item \textbf{N6}, \textbf{N8} und \textbf{N9} durch Espressotests
	\item \textbf{N7} durch das Nicht-Implementieren der Funktion
\end{itemize}
Die nicht genannten Nichtfunktionalen Anforderungen sind nicht implementiert.
