\chapter{Zielbestimmung}

Dieses Dokument bietet ein Bericht zur Qualitätssicherung des Projektes "`Exzellenzkoch"'. Wie bereits in vorigen Dokumenten wird sich an den drei Schichten der zugrundeliegenden Clean Architecture orientiert:
\begin{itemize}
	\item User Interface Layer
	\item Data Layer
	\item Domain Layer
\end{itemize}
Die dokumentierten Testfälle beinhalten alle im Pflichtenheft spezifizierten Testfälle, die zum derzeitigen Implementierungsstand umsetzbar sind. Es werden also alle Muss-, sowie implementierte Wunschkriterien getestet.

\section{Paketübersicht}
\subsection{Client}
Generell teilen sich die Testpakete in zwei Stränge auf: Unittests und andere Tests.
Unittests sind im Paket \textbf{de.psekochbuch.exzellenzkoch} unter "`test"' zu finden und andere Tests in \textbf{de.psekochbuch.exzellenzkoch} unter "`androidTest"'. 
Unter "`andere Tests"' zählen sowohl die Tests des UI Layers, sowie auch die Repository- und Lasttests.
Die Tests des User Interface Layers befinden sich im Paket \textbf{de.psekochbuch.exzellenzkoch.testcases}, sowie in \textbf{de.psekochbuch.exzellenzkoch.navigation}.
Die Tests des Data Layers teilen sich auf in Servertests und Tests der Interface Implementierungen.
Repositorytests befinden sich im Paket
\textbf{de.psekochbuch.exzellenzkoch.datalayer}.
Der Server wird in seinem eigenen Repo getestet.
Lastentests sind im Paket \textbf{de.psekochbuch.exzellenzkoch.lastentest} zu finden.
Das Domain Layer enthält lediglich die Domain Entities, sowie die Interfaces, welche nicht getestet werden.

\subsection{Server}
Servertests befinden sich im Paket \\
\textbf{package de.psekochbuch.exzellenzkoch} unter "`test.kotlin"'