\newglossaryentry{pRezept}
{
	name=privates Rezept,
	plural=private Rezepte,
	description={Rezept was ein Nutzer für sich in seiner Rezeptliste gespeichert hat}
}

\newglossaryentry{oRezept}
{
	name=veröffentlichtes Rezept,
	plural=veröffentlichte Rezepte,
	description={Rezept was veröffentlicht wurde, muss zusätzliche Bedingungen erfüllen: Beispielsweise darf der Titel nicht leer sein}
}

\newglossaryentry{Kochbuchserver}
{
	name=Kochbuchserver,
	plural=Kochbuchserver,
	description={der Serverteil der beispielsweise Rezepte und Profile speichert und persistiert. Er kommuniziert über das Internet mit der Kochbuchapp}
}

\newglossaryentry{Kochbuchapp}
{
	name=Kochbuchapp,
	plural=Kochbuchapps,
	description={die Android basierte App des Kochbuchs}
}

\newglossaryentry{Dialogfenster}
{
	name=Dialogfenster,
	plural=Dialogfenster,
	description={Fenster, das als Teil einer grafischen Benutzeroberfläche über dem Anwendungsfenster erscheint, um eine klar abgegrenzte Aufgabe zu erfüllen}
}

\newglossaryentry{ID}
{
	name=ID,
	plural=IDs,
	description={Bei uns wird hiermit die Benutzerid gemeint, das ist ein Name der den Nutzer eindeutig identifiziert. Er ist der Name der im Profil der Nutzer angezeigt wird. Außerdem wird er für die interne Appkommunikation, um Beispiel um alle veröffentlichte Rezepte eines Nutzers beim Server anzufragen, benötigt}
}

\newglossaryentry{Profilansicht}
{
	name=Profilansicht,
	plural=Profilansichten,
	description={Eine private Ansicht, die einem Nutzer alle relevanten Informationen zu seinem Profil bietet (ID, Email, Profilbild, Passwort (nicht im Klartext))}
}

\newglossaryentry{IPD}
{
	name=Institut für Programmstrukturen und Datenorganisation,
	plural=Institute für Programmstrukturen und Datenorganisation,
	description={Das Institut ist diejenige Einrichtung der Fakultät, die sich in Forschung und Lehre mit der Software-Technik als Ingenieursdisziplin befasst}
}

\newglossaryentry{KIT}
{
	name=Karlsruher Institut fuer Technologie,
	plural=Karlsruher Institut fuer Technologie,
	description={Forschungsuniversität in Karlsruhe}
}

\newglossaryentry{REST}
{
	name=REST,
	plural=REST,
	description={REST bezeichnet einen Architekturstil Webservices und deren Schnittstellen zu programmieren}
}

\newglossaryentry{Feed}
{
	name=Feed,
	plural=Feed,
	description={Ein Feed informiert den Nutzer über Veränderungen auf einer Webseite oder einem Profil, das abonniert wurde}
}

\newglossaryentry{Android}
{
	name=Android,
	plural=Android,
	description={Android ist sowohl ein Betriebssystem als auch eine Software-Plattform für mobile Geräte wie Smartphones, Mobiltelefone, Fernseher, Mediaplayer, Netbooks und Tabletcomputer das von Google und der von Google gegründeten Open Handset Alliance entwickelt wird}
}

\newglossaryentry{Smartwatch}
{
	name=Smartwatch,
	plural=Smartwatches,
	description={Eine Smartwatch ist eine elektronische Armbanduhr, die über zusätzliche Sensoren, Aktuatoren sowie Computerfunktionalitäten und -konnektivitäten verfügt}
}

\newglossaryentry{öAktivität}
{
	name=öffentliche Aktivität,
	plural=öffentliche Aktivitäten,
	description={Die öffentliche Äktivität eines Nutzers bezeichnet seine öffentlichen Aktionen in der Applikation: ein neues Rezept, oder einen Kommentar veröffentlichen und ein öffentliches Rezept bearbeiten}
}

\newglossaryentry{Nutzer}
{
	name=Nutzer,
	plural=Nutzer,
	description={Nutzer, der die Applikation entweder nicht eingeloggt oder angemeldet nutzt}
}

\newglossaryentry{privater Nutzer}
{
	name=privater Nutzer,
	plural=private Nutzer,
	description={Nutzer, der die Applikation privat, also ohne angemeldet zu sein nutzt}
}

\newglossaryentry{angemeldeter Nutzer}
{
	name=angemeldete Nutzer,
	plural=angemeldete Nutzer,
	description={Nutzer, der in der Applikation mit einem Account angemeldet ist}
}

\newglossaryentry{Autor}
{
	name=Autor,
	plural=Autoren,
	description={Ersteller eines oder mehrerer Rezepte. Sowohl ohne, als auch mit Account möglich}
}

\newglossaryentry{Suchfilter}
{
	name=Suchfilter,
	plural=Suchfilter,
	description={Angaben, die die Suche durch die Rezept-Datenbank einschränken}
}

\newglossaryentry{Suchvorschlag}
{
	name=Suchvorschlag,
	plural=Suchvorschläge,
	description={Bei der Eingabe von Suchfiltern angezeigt Wörter, nach denen gesucht werden kann}
}

\newglossaryentry{Tag}
{
	name=Tag,
	plural=Tags,
	description={Label, welches eine Eigenschaft trägt, zB {\glqq scharf\grqq}, {\glqq deftig\grqq}, {\glqq indisch\grqq}, {\glqq süß\grqq}, etc}
}

\newglossaryentry{bearbeiten}
{
	name=bearbeiten,
	plural=bearbeiten,
	description={Den Inhalt eines Objekts verändern, insbesondere auch löschen}
}

\newglossaryentry{lokal}
{
	name=lokal,
	plural=lokal,
	description={Nur den jeweiligen Anwender betreffend, ohne globale Auswirkungen}
}
\newglossaryentry{Template}
{
	name=Template,
	plural=Templates,
	description={Template englisch für Schablone oder Vorlage}
}

\newglossaryentry{Rezeptliste}
{
	name=Rezeptliste,
	plural=Rezeptlisten,
	description={Eine Liste von Rezepten, die der Nutzer verwalten kann, die eigene vollständige und unvollständige Rezepte beinhaltet}
}
\newglossaryentry{Favoritenliste}
{
	name=Favoritenliste,
	plural=Favoritenlisten,
	description={Eine Liste von Rezepten, die der Nutzer verwalten kann. Der Nutzer kann veröffentlichte Rezepte dieser Liste hinzufügen und entfernen}
}
\newglossaryentry{Zubereitungszeit}
{
	name=Zubereitungszeit,
	plural=Zubreitungszeiten,
	description={Die Zeit, die man zum Schnippeln, Rühren und generellen Vorbereiten der Zutaten braucht}
}
\newglossaryentry{Backzeit/Schmorzeit}
{
	name=Backzeit/Schmorzeit,
	plural=Backzeiten/Schmorzeiten,
	description={Die Zeit, die das Gericht nach der Zubereitungszeit noch braucht. Beispiel: ein Kuchen Bäckt noch eine Stunde, oder Rinderrouladen müssen noch zwei Stunden schmoren}
}
\newglossaryentry{Freund}{
	name={Freund},
	plural={Freunde},
	description={Nutzer können sich in der Applikation als Freunde hinzufügen und somit leichter untereinander Rezepte austauschen und kommunizieren}
}
\newglossaryentry{Freundesgruppe}{
	name={Freundesgruppe},
	plural ={Freundesgruppen},
	description={Gruppen, die es Freunden ermöglichen innerhalb der App zu kommunizieren}
}
