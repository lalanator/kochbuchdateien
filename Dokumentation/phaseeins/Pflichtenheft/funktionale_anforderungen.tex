\chapter{Funktionale Anforderungen}


	 
	%Rezeptverwaltung
\section{Rezeptverwaltung}

	\subsection{Erstellung von Rezepten}
	
	\functionality{Neues Rezept erstellen}{fnc:rec_new}
		\Glspl{Nutzer} können neue Rezepte erstellen
	\fulfills{crt:rezepterstellen}
	
		\functionality{Bild hinzufügen}{fnc:rec_pic}
			\Glspl{Autor} können ihren Rezepten ein Bild hinzufügen
		\fulfills{crt:rezepterstellen}

		\functionality{Titel}{fnc:rec_title}
			\Glspl{Autor} können ihren Rezepten einen Titel geben
		\fulfills{crt:rezepterstellen}

		\functionality{Zutatenliste}{fnc:rec_zutaten}
			\Glspl{Autor} können in ihren Rezepten eine Zutatenliste angeben
		\fulfills{crt:rezepterstellen}

		\functionality{Mengenangabe}{fnc:rec_mengen}
			\Glspl{Autor} können zu jeder Zutat einer Zutatenliste Mengenangaben machen
		\fulfills{crt:rezepterstellen}

		\functionality{Zubereitungsbeschreibung}{fnc:rec_beschreib}
			\Glspl{Autor} können ihren Rezepten eine Zubereitungsbeschreibung hinzufügen
		\fulfills{crt:rezepterstellen}

		\functionality{Tags}{fnc:rec_tags}
			\Glspl{Autor} können ihren Rezepten \glspl{Tag} aus einer vordefinierten Liste hinzufügen
		\fulfills{crt:rezepterstellen}

		\functionality{Gesamtzeit aufteilen}{fnc:rec_time_split} 
			\Glspl{Autor} können ihren Rezepten Zeiten zuordnen:
					\begin{itemize}[nosep]
						\item \gls{Zubereitungszeit}
						\item \gls{Backzeit/Schmorzeit}
					\end{itemize}
		\fulfills{crt:rezepterstellen}
		
		\functionality{Gesamtzeit}{rec_time}
			Aus \gls{Zubereitungszeit} und der \gls{Backzeit/Schmorzeit} wird die Gesamtzeit errechnet
		\fulfills{crt:rezepterstellen}
					
		\functionality{Portionenanzahl}{fnc:rec_portions}
			\Glspl{Autor} können die Anzahl an Portionen, die eines ihrer Rezepte bedient, festlegen
		\fulfills{crt:rezepterstellen}

		
	\subsection{Speichern von Rezepten}
	
		\functionality{Automatisches Speichern}{fnc:rec_save}
			\Glspl{pRezept} werden automatisch in der \gls{Rezeptliste} gespeichert, auch wenn sie noch nicht fertig bearbeitet sind
		\fulfills{crt:speichere_unvollst}
		
		\functionality{Rezeptliste}{fnc:rec_list}
			Die \gls{Rezeptliste} hat einen eigenen Menüpunkt im Appmenü
		\fulfills{crt:speichere_unvollst}
		
	\subsection{Veröffentlichen von Rezepten}
	
		\functionality{Autoren können ihre privaten Rezepte veröffentlichen}{fnc:rec_pub}
			Bei Veröffentlichung muss das Rezept:
				\begin{itemize}[nosep]
					\item einen Titel haben
					\item eine Zutatenliste haben
					\item eine Zubereitungsbeschreibung haben
					\item eine Zubereitungsdauer haben
					\item eine Anzahl an Portionen haben
				\end{itemize}
  	\fulfills{crt:rezeptveroeffentlichen}
		\fulfills{crt:rezeptdurchsuchen}
	
		\functionality{Erstelldatum}{fnc:rec_date}
		 Dem Rezept wird bei Veröffentlichung das Erstelldatum hinzugefügt
		\fulfills{crt:rezeptveroeffentlichen}
		\fulfills{crt:rezeptdurchsuchen}
		
		
	\subsection{Bearbeitung von Rezepten}
	
	\functionality{Private Rezepte bearbeiten}{fnc:rec_edit}
		\glspl{Autor} können alle Felder ihrer privaten Rezepten \gls{bearbeiten}
	\fulfills{crt:rezeptverwalten}
		
	\functionality{Öffentliche Rezepte bearbeiten}{fnc:rec_pub_edit}
	Wenn Autoren das zugehörige private Rezept eines öffentlchen Rezepts erneut veröffentlichen, ist das veröffentlichte Rezept in der neuen bearbeiteten Version verfügbar
	\fulfills{crt:rezeptverwalten}
	
	\functionality{Rezepte löschen}{fnc:rec_del}
		\glspl{Autor} können ihre privaten Rezepte und veröffentlichten Rezepte löschen
	\fulfills{crt:rezeptverwalten}


	\subsection{Favoriten}
	
	\functionality{Favorisieren von Rezepten}{fnc:fav}
	Nutzer können \glspl{oRezept} als Favoriten auswählen
	\fulfills{crt:rezeptdurchsuchen}
	\fulfills{crt:rezeptfav}

	\functionality{Favoritenliste}{fnc:fav_men}
	Favorisierte Rezepte haben einen eigenen Menüpunkt im Appmenü
	\fulfills{crt:rezeptfav}
	
	
	\subsection{{\em optional} Mengenangaben skalieren}
	
	\functionality{Portionenanzahl skalieren}{fnc:port_skal}
		Die Portionenanzahl eines Rezepts ist von einem Nutzer in der Rezeptanzeige änderbar.
		Die Zutatenmengen in der Anzeige passen sich entsprechend an
	\fulfills{crt:mengenangaben_skalieren}
	
	\subsection {{\em optional} Einkaufsliste}	
	\functionality{Einkaufsliste}{fnc:shop_who}
		Alle \glspl{Nutzer} haben eine Einkaufsliste. Diese kann leer sein
	\fulfills{crt:einkaufsliste}
	
	\functionality{Zutaten hinzufügen}{fnc:shop_add}
		\glspl{Nutzer} können Zutaten eines Rezepts auswählen und ihrer Einkaufsliste hinzufügen
	\fulfills{crt:einkaufsliste}
	
	\functionality{Mengenangaben übernehmen}{fnc:shop_men}
		Die eingestellten Mengenangaben einer Zutat werden mit in die Einkaufsliste übernommen
	\fulfills{crt:einkaufsliste}
	
	\functionality{Einkaufsliste bearbeiten}{fnc:shop_del}
		\glspl{Nutzer} können Zutaten von ihrer Einkaufsliste löschen
	\fulfills{crt:einkaufsliste}
	
	\functionality{Einkaufsliste finden}{fnc:shop_pos}
		Die Einkaufsliste hat einen eigenen Menüpunkt im Appmenü
	\fulfills{crt:einkaufsliste}


	\subsection{{\em optional} Import- und Exportfunktion}
		
		\functionality{Export}{fnc:exp}
		Rezepte können in eine Datei \gls{lokal} auf dem Gerät  \gls{Nutzer}s exportiert, bzw. an eine andere App, wie Notizapp oder Mailapp per Teilen an-Knopf gesendet werden
		\fulfills{crt:importexport}
		
		\functionality{Import}{fnc:imp}
		Rezepte können aus externen Quellen als neue \glspl{pRezept} in die Applikation geladen werden. Externe Quellen sind:
		\begin{itemize}[nosep]
					\item Rezepte von bereits existierenden Kochwebseiten
					\item Text aus anderen Apps
				\end{itemize}
		\fulfills{crt:importexport}

	
	\subsection{{\em optional} Textverweise auf interne Rezepte}
	
		\functionality{Rezepte verlinken}{fnc:link}
		\glspl{Autor} können in ihrem Rezept einzelne Wörter auswählen und ihnen pro Wort ein \glspl{oRezept} zuordnen
		\fulfills{crt:rezeptverlinken}
		
		
		\functionality{Anzeige von Verlinkungen}{fnc:linkcol}
		Verlinkte Wörter werden nach Veröffentlichung des Rezepts farblich vom restlichen Text abgehoben
		\fulfills{crt:rezeptverlinken}
		
		\functionality{Aufrufen von Verlinkten Rezepten}{fnc:linkact}
		\Glspl{Nutzer} können verlinkte Rezepte durch das Auswählen eines markierten Wortes anzeigen lassen
		\fulfills{crt:rezeptverlinken}
		
		\functionality{Inaktive Verlinkungen}{fnc:linkinact}
		Ist ein \gls{oRezept} nicht mehr vorhanden (zB gelöscht), aber noch in einem anderen Rezept verlinkt, wird eine Fehlermeldung angezeigt
		\fulfills{crt:rezeptverlinken}
		
		\subsection{{\em optional} Rezeptstruktur}
	  \functionality{Eingabe von Überschriften}{fnc:ueingabe}
	  Nutzer können die Zutatenliste und Rezepteingabe durch Überschriften untergliedern
		\fulfills{crt:rezeptstruktur}
		
		\functionality{Anzeige von Überschriften}{fnc:uanzeige}
		In der Rezeptanzeige wird die Zutatenliste und Rezepteschreibung mit Überschriften zu einzelnen Abschnitten   formatiert dargestellt 
		\fulfills{crt:rezeptstruktur}
		

	\subsection{{\em optional} Bewertungsfunktion}
		
		\functionality{Bewertung}{fnc:bew}
		\Glspl{angemeldeter Nutzer} können jedem veröffentlichten Rezept einen Wert von 1 bis 5 zuordnen
		\fulfills{crt:bewerten}
		

	\subsection{{\em optional} Kommentarfunktion}

		\functionality{Kommentare anzeigen}{fnc:com_see}
		Unter einem veröffentlichten Rezept werden vorhandene Kommentare angezeigt
		\fulfills{crt:rezept_kommentieren}
	
	
		\functionality{Rezept kommentieren}{fnc:com_wr}
		\Glspl{angemeldeter Nutzer} können einem \gls{oRezept} Kommentare hinzufügen
		\fulfills{crt:rezept_kommentieren}
		
		\functionality{Kommentare bearbeiten}{fnc:com_ed}
		Nutzer können ihre Kommentare unter einem veröffentlichten Rezept \gls{bearbeiten}
		\fulfills{crt:rezept_kommentieren}
		
%Benutzerverwaltung
\section{Benutzerverwaltung}

	\subsection{Profilerstellung}
	
	\functionality{Profilerstellung}{fnc:prf}
	\Glspl{Nutzer} müssen bei der Erstellung folgende Daten angeben: eindeutige E-Mail-Adresse und Passwort. \Glspl{angemeldeter Nutzer} können eine eindeutige \gls{ID} angeben. 
	\fulfills{crt:make_acc}
	
	\functionality{Benutzer ID zuweisen}{fnc:id}
	Falls \Glspl{Nutzer} keine eindeutige \gls{ID} bei Profilerstellung angibt, wird ihm eine eindeutige \gls{ID} zugewiesen
	\fulfills{crt:make_acc}	
	
		
	\subsection{Profilbearbeitung}
	
	\functionality{Passwort ändern}{fnc:pw_chg}
	\Glspl{angemeldeter Nutzer} können eigenes Passwort ändern
	\fulfills{crt:make_acc}
	
	\functionality{Benutzer-Id ändern}{fnc:id_chg}
	\Glspl{angemeldeter Nutzer} können ihre \gls{ID} ändern
	\fulfills{crt:make_acc}

	
	\functionality{Profil löschen}{fnc:del_profil}
	\Glspl{angemeldeter Nutzer} können eigenes Profil löschen. Dabei werden alle Benutzerdaten, außer gespeicherte Rezepte, gelöscht
	\fulfills{crt:make_acc}
		
	\subsection{{\em optional} Benutzerprofil}
	
	\functionality{Personenbeschreibung hinzufügen}{fnc:prsn_dsc}
	\Glspl{angemeldeter Nutzer} können eine textuelle Kurzbeschreibung seiner selbst zum Benutzerprofil hinzufügen
	\fulfills{crt:profilansicht}

	\functionality{Personenbeschreibung bearbeiten}{fnc:prsn_chg}
	\Glspl{angemeldeter Nutzer} können eigene Personenbeschreibung bearbeiten
	\fulfills{crt:profilansicht}

	\functionality{Profilbild hinzufügen}{fnc:prsn_dscc}
	\Glspl{angemeldeter Nutzer} können ein Profilbild hinzufügen. Falls kein Profilbild hinzugefügt wurde, wird Default-Bild eingestellt
	\fulfills{crt:account_profilbild}
	
	\functionality{Profilbild bearbeiten}{fnc:prsn_chgg}
	\Glspl{angemeldeter Nutzer} können eigenes Profilbild \glspl{bearbeiten}
	\fulfills{crt:account_profilbild}
	
	\functionality{Profil anschauen}{fnc:prfl_rprsnt}
	\Glspl{Nutzer} können Profile von anderen angemeldeten Nutzern anschauen. Dabei sind ID, Liste veröffentlichter Rezepte, Personenbeschreibung und Profilbild sichtbar
	\fulfills{crt:profilansicht}
	
	\functionality{Profil abonnieren}{fnc:prfl_abb}
	\Glspl{angemeldeter Nutzer} müssen andere \glspl{angemeldeter Nutzer} abonnieren und wieder deabonnieren können
	\fulfills{crt:profilansicht}
	
	\functionality{Abonnierte Profile}{fnc:wtch_abb}
	Abonnierte, \glspl{angemeldeter Nutzer} sind in einem eigenen Menüpunkt im Appmenü einsehbar
	\fulfills{crt:profilansicht}
	

	\subsection{Nutzeranmeldung}
	
	\functionality{Anmeldung}{fnc:usr_aut}
	Zum Login werden \gls{ID} und zugehöriges Passwort benutzt
	\fulfills{crt:login_acc}
	

%Suche	
\section{Suche}

	\subsection{Rezeptsuche}
	
		\functionality{Suche}{fnc:rcp_srch}
		\Glspl{Nutzer} können nach öffentlichen Rezepten suchen und mehrere \glspl{Suchfilter} einstellen und anwenden
		\fulfills{crt:rezeptdurchsuchen}
		
		\functionality{Suchfilter}{fnc:rcp_fltr}
		\Glspl{Nutzer} können Rezepttitel, Zutaten, \glspl{Tag} und Zubereitungsdauer als \glspl{Suchfilter} einstellen
		\fulfills{crt:rfilter}

			
	\subsection{Ergebnisse Rezeptsuche}
	
		\functionality{Ergebnis}{fnc:srch_rslt}
		Alle durch die \gls{Suchfilter} gefundenen öffentlichen Rezepte werden untereinander angezeigt
		\fulfills{crt:rezeptdurchsuchen}
		
		\functionality{Rezeptanzeige}{fnc:rcp_rprsnt}
		Die aufgelisteten Rezepte beinhalten den Titel und das Bild des Rezepts
		\fulfills{crt:rezeptdurchsuchen}
		
		%wie wäre noch Dauer? --als Wunschkrit. vllt? 
	
		\functionality{Sortierung}{fnc:srch_sort}
		Die Suchergebnisse sind nach Durchschnittsbewertung, Erstellungsdatum oder Favorisierung sortierbar
		\fulfills{crt:rezeptdurchsuchen}
	
	
	\subsection{{\em optional} Erweiterte Suche}
		
		\functionality{Bewertungssuche}{fnc:srch_rtng}	
		Bewertungen ist ein \gls{Suchfilter}
		\fulfills{crt:erw_suche}
		
		\functionality{Zubereitungsbeschreibung}{fnc:srch_prep}
		Zubereitungsbeschreibung ist ein \gls{Suchfilter}
		\fulfills{crt:erw_suche}
		
		\functionality{Sortierung}{fnc:srch_sortt}
		Die Suchergebnisse sind nach durchschnittlicher Bewertung der Suchergebnisse sortierbar
		\fulfills{crt:rezept_bewertung_suchen}
	
	
	\subsection{{\em optional} Benutzersuche}
	
		\functionality{Benutzersuche}{fnc:usr_srch}
		\Glspl{Nutzer} können nach \glspl{angemeldeter Nutzer} suchen und dabei mehrere \gls{Suchfilter} einstellen und anwenden
		\fulfills{crt:user_search}
		
		\functionality{Suchfilter}{fnc:usr_srch_filter}
		Die \gls{ID} ist ein \gls{Suchfilter}
		\fulfills{crt:user_search}
		
		\functionality{Benutzeranzeige}{fnc:fnc_usr_rprsnt}
		Die Suchergebnisse \glspl{angemeldeter Nutzer} beinhalten \glspl{ID}
		\fulfills{crt:user_search}
		
		\functionality{Oberfläche Benutzersuche}{fnc:fnc_usr_rprsntt}
		Die Nutzersuche kriegt einen eigenen Menüpunkt im Appmenü
		\fulfills{crt:user_search}
		
		
		\subsection{{\em optional} Freundesystem}
	
	\functionality{Freund anfragen}{fnc:add_frnd}
	\Glspl{angemeldeter Nutzer} können anderen \glspl{angemeldeter Nutzer} eine Freundschaftsanfrage schicken
	\fulfills{crt:account_freunde}
	
	\functionality{Freundschaftsanfragen}{fnc:frnd_rqst}
	\Glspl{angemeldeter Nutzer} können Freundschaftsanfragen annehmen oder ablehnen
	\fulfills{crt:account_freunde}
	
	\functionality{Freunde löschen}{fnc:frnd_del}
	\Glspl{angemeldeter Nutzer} können \glspl{Freund} löschen
	\fulfills{crt:account_freunde}

	\functionality{Freunde anzeigen}{fnc:frnd_rprsnt}
	\Glspl{angemeldeter Nutzer} können alle \glspl{Freund} anzeigen
	\fulfills{crt:account_freunde}

	\functionality{Gruppenmitglieder}{fnc:grp_add}
	\Glspl{angemeldeter Nutzer} können Freunde zu einer eigenen Freundesgruppe hinzufügen und entfernen
	\fulfills{crt:account_freunde}

		
	\functionality{Gruppen}{fnc:frnd_grp}
	\Glspl{angemeldeter Nutzer} können eigene Freundesgruppe erstellen und diese löschen
	\fulfills{crt:account_freunde}

	\Glspl{angemeldeter Nutzer} können \glspl{Freund} zu einer eigenen \gls{Freundesgruppe} hinzufügen und entfernen
	

	\functionality{Rezepte teilen}{fnc:frnd_rcp}
	\Glspl{angemeldeter Nutzer} können \glspl{pRezept} an \glspl{Freund} oder an \glspl{Freundesgruppe}, in denen sie Mitglied sind, schicken
	\fulfills{crt:account_freunde}
		
		
	\subsection{{\em optional} Rezepte Feed}
	
		\functionality{Feed}{fnc:rcp_fd}
		Beim Öffnen der App wird ein \gls{Feed} mit den neusten veröffentlichten Rezepten angezeigt 
		\fulfills{crt:rezept_feed}
		
		
		\functionality{Feed Menü}{fnc:rcp_fd_m}
		Der \gls{Feed} kann durch einen Menüpunkt "`Startseite"' aufgerufen werden
		\fulfills{crt:rezept_feed}
	
		\functionality{Feed Menü}{fnc:rcp_fd_act}
		Wird der \gls{Feed} erneut aufgerufen, zeigt er die aktuell neusten \glspl{oRezept}
		\fulfills{crt:rezept_feed}
		
	\subsection{{\em optional} Private Tags}
			
		\functionality{Tags hinzufügen}{fnc:tags_wk}
			Jeder Nutzer kann der Liste an Tags welche hinzufügen, die nur er sieht und suchen kann
		\fulfills{crt:wktags}
		


	%Kochbuchserver
\section{Kochbuchserver}
Damit die definierten Wunsch- und Musskriterien umgesetzt werden können muss der Server einige Funktionen in seiner
Schnittstelle bereitstellen, die im Folgenden spezifiziert sind:

	\functionality{Speichern eines neuen privaten Rezepts}{fnc:server_create}
	\fulfills{crt:rezeptveroeffentlichen}
	\fulfills{crt:speichere_unvollst}


  \functionality{Ändern eines privaten Rezepts}{fnc:server_update}
	\fulfills{crt:rezeptverwalten}

	\functionality{Löschen eines privaten Rezepts}{fnc:server_delete}
	\fulfills{crt:rezeptverwalten}

	\functionality{Veröffentlichen eines privaten Rezepts}{fnc:ser_pr_ver}
  Mit Aufruf dieser Funktion wird ein privates Rezept veröffentlicht und ist als \glspl{oRezept} 
  verfügbar.
	Falls es schon veröffentlicht war, wird durch den Aufruf das öffentliche Rezept mit aktuellen
 Daten des privaten Rezepts aktualisiert
 
	\fulfills{crt:rezeptveroeffentlichen}


	\functionality{Laden eines privaten Rezepts}{fnc:server_read}
	Ein privates Rezept für eine Funktion laden
	\fulfills{crt:rezeptverwalten}

	\functionality{Laden eines öffentlichen Rezepts}{fnc:server_read}
	Ein öffentliches Rezept für eine Funktion laden
	\fulfills{crt:rezeptdurchsuchen}


	\functionality{Entfernen eines öffentlichen Rezepts}{fnc:server_delete_o}
  Mit dem Aufruf dieser Funktion wird das öffentliche Rezept entfernt
	\fulfills{crt:rezeptverwalten}
	

  \functionality{Laden der Profildaten eines Nutzers}{fnc:userdata}
  Lädt öffentlich verfügbare Profildaten eines \gls{Nutzer}s
	\fulfills{crt:account_profilbild}
	\fulfills{crt:login_acc}
  
  \functionality{Ändern der Profildaten des eigenen Nutzers}{fnc:server_chg_userdata}
  Erlaubt, änderbare Einheiten des Benutzerprofils zu ändern
%  Stellt sicher dass die Änderungen konsistent sind. Zum Beispiel wird bei den öffentlichen Rezepten
% nach Änderung der Benutzerid, die Benutzerid des Authors angepasst. 
% Anmerkung: Zu schwammig für FA "`konsistent"' zB - lieber raus
	\fulfills{crt:make_acc}
  
  \functionality{Laden der Liste der privaten Rezepte einer Benutzerid}{fnc:server_delete}
	\fulfills{crt:speichere_unvollst}
   
  \functionality{Laden der Tags des eigenen Benutzers}{fnc:wk_tag}
   \fulfills{crt:wktags}
	
  \functionality{Speichern der Tags des eigenen Benutzers}{fnc:wk_tags}
	\fulfills{crt:wktags}
  
  \functionality{Laden der Einkaufsliste des eigenen Benutzers}{fnc:server_ld_ek}
	\fulfills{crt:einkaufsliste}
   
  \functionality{Speichern der Einkaufsliste des eigenen  Benutzers}{fnc:server_sv_ek}
	\fulfills{crt:einkaufsliste}

  \functionality{Rezeptsuche}{fnc:server_rec_search}
Gibt eine Liste mit Rezept-IDs zurück von öffentlichen Rezepten die den Suchfilterkriterien entsprechen. Liste gegebenenfalls sortiert ausgeben
	\fulfills{crt:rezeptdurchsuchen}
  
   \functionality{Benutzersuche}{fnc:server_ben_search}
 Gibt eine Liste mit Benutzer-IDs von Benutzern zurück, die den Suchfilterkriterien entsprechen
	\fulfills{crt:user_search}

  \functionality{Speichern eines Kommentars}{fnc:server_kom_sv}
   Speichert einen zu einer bestimmten Rezept-ID eines öffentlichen Rezeptes gehörigen Kommentar.
   Wird die Funktion auf dem gleichen Kommentar mehrmals aufgerufen, wird der alte Kommentar mit dem neuen Text überschrieben
	\fulfills{crt:rezept_kommentieren}
      
  \functionality{Speichern eines neuen Freundes}{fnc:server_fr_sv}
  Speichert zu einem Nutzer einen Freund, der auch ein Nutzer ist, falls dieser die Anfrage annimmt
  \fulfills{crt:account_freunde}
  
  \functionality{Löschen eines Freundes}{fnc:server_fr_dl}
  Von einem Nutzer wird ein Freund gelöscht
  \fulfills{crt:account_freunde}
  
  \functionality{Speichern einer Freundesgruppe}{fnc:server_fg_sv}
  Erstellte Freundesgruppe mit Verweisen zu Freunden und Rezepte speichern
  \fulfills{crt:account_freunde}
  
  \functionality{Ändern einer Freundesgruppe}{fnc:server_fg_ed}
  Die Verweise zu den Rezepten und Freunden wird aktualisiert
  \fulfills{crt:account_freunde}
  
  \functionality{Löschen einer Freundesgruppe}{fnc:server_fg_dl}
  Die Freundesgruppe wird gelöscht und die Verweise aufgelöst
  \fulfills{crt:account_freunde}
  
  
  \section{Admin}
  \functionality {gemeldete Rezepte löschen}
  der Admin kann Rezepte mit unangemessenen Inhalt manuell prüfen und entfernen
  
  \functionality {gemeldete Gruppen löschen}
  der Admin kann Gruppen mit unangemessenen Inhalt manuell prüfen und entfernen
  
  \functionality {gemeldete Profile verwalten und löschen} der Admin kann Profile mit unangemessen Inhalten manuell prüfen und entfernen
  
  Der Admin ist verantwortlich dafür gemeldete Rezepte auf ihre Angemessenheit zu überprüfen und kann diese Entfernen. Ist ein Profil nicht in Ordnung, löscht der Admin dieses über die Firebase Datenbank. 
