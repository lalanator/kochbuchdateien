\chapter{Verwendete Bibliotheken}

\section{App}

\subsection{Android Jetpack}
Aus der Android Jetpack Library werden ViewModel, Lifedata und Room verwendet. Lifedata ist für das Binding zwischen verschiedenen UI Komponenten zuständig und Room ist für das Aufsetzen und Verwalten der lokalen Datenbank nötig.

\subsection{Glide}
Um Bilder in die Fragments zu laden, wird \href{https://github.com/bumptech/glide}{Glide} benutzt.

\subsection{Hdodenhof}
Um Bilder in ImageViews in der UI kreisförmig anzeigen zu können, wird die \href{https://github.com/hdodenhof/CircleImageView}{CircleImageView} Bibliothek benutz.

\subsection{Retrofit}
Retrofit wird genutzt, um die über HTTP commands mit dem Server zu kommunizieren.


\section{Server}

\subsection{Firebase}
Firebase wird verwendet, um die Nutzerauthentifizierung umzusetzen. Firebase wird auch in der \textbf{App} genutzt.

\subsection{Springboot}
Der Server basiert auf SpringBoot.

\subsection{MariaDB}
Um die Serverdatenbank mit SQL zu modellieren, wird MariaDB genutzt.

\subsection{Commons-IO}
Für das File-Handling im Server wird die Apache Commons IO Library genutzt.